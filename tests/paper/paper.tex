\documentclass[article]{jss}

%%%%%%%%%%%%%%%%%%%%%%%%%%%%%%
%% declarations for jss.cls %%%%%%%%%%%%%%%%%%%%%%%%%%%%%%%%%%%%%%%%%%
%%%%%%%%%%%%%%%%%%%%%%%%%%%%%%

%% almost as usual
\author{Marco Colombo,
        Felix Agakov,
        Paul McKeigue\\
        University of Edinburgh}
\title{Cross-Validated (Nested) Forward Selection:\\
       the \proglang{R} Package \pkg{nestfs}}

%% for pretty printing and a nice hypersummary also set:
\Plainauthor{Achim Zeileis, Second Author} %% comma-separated
\Plaintitle{Cross-Validated (Nested) Forward Selection: the R Package nestfs} %% without formatting
\Shorttitle{\pkg{nestfs}: Cross-Validated (Nested) Forward Selection} %% a short title (if necessary)

%% an abstract and keywords
\Abstract{
Forward selection is a well-know technique for constructing a predictive
model that has the properties of sparsity and interpretability. As for other
techniques used to solve a prediction problem, it can be easily mis-used
if its parameters are learned outside of a cross-validation setting, thus
producing over-optimistic estimates and models that do not generalise well
in prediction of withdrawn data.

In this paper we present an implementation of forward selection for the
\proglang{R} programming language based on linear and logistic regression
which adopts cross-validation as a core
component of the selection procedure. The \pkg{nestfs} package features
several selection and termination criteria, and is parallelised over the
cross-validation folds.
}
\Keywords{forward selection, cross-validation, \proglang{R}}
\Plainkeywords{forward selection, cross-validation, R} %% without formatting
%% at least one keyword must be supplied

%% publication information
%% NOTE: Typically, this can be left commented and will be filled out by the technical editor
%% \Volume{50}
%% \Issue{9}
%% \Month{June}
%% \Year{2012}
%% \Submitdate{2012-06-04}
%% \Acceptdate{2012-06-04}

%% The address of (at least) one author should be given
%% in the following format:
\Address{
  Marco Colombo \\
  Centre for Population Health Sciences\\
  University of Edinburgh \\
  Medical School \\
  Teviot Place \\
  Edinburgh \\
  EH8 9AG \\
  E-mail: \email{m.colombo@ed.ac.uk}\\
  URL: \url{http://maths.ed.ac.uk/~mcolombo/}
}
%% It is also possible to add a telephone and fax number
%% before the e-mail in the following format:
%% Telephone: +43/512/507-7103
%% Fax: +43/512/507-2851

%% for those who use Sweave please include the following line (with % symbols):
%% need no \usepackage{Sweave.sty}

%% end of declarations %%%%%%%%%%%%%%%%%%%%%%%%%%%%%%%%%%%%%%%%%%%%%%%


\begin{document}

%% include your article here, just as usual
%% Note that you should use the \pkg{}, \proglang{} and \code{} commands.

%\section[About Java]{About \proglang{Java}}
%% Note: If there is markup in \(sub)section, then it has to be escape as above.

\section{Literature review}

Should describe the history of forward selection: when and why it was
introduced, what sort of problems it was originally designed to solve.

Talk about techniques that build on it or are comparable to it (stepwise
selection, backward elimination). Mention other types of parametric or
non-parametric ways of building a predictive model.

Talk about known advantages (interpretability, sparsity) and disadvantages
(speed) compared to lasso-type models.

Talk about types of problems for which forward selection struggles: large
number of predictors, correlated predictors. Talk about how these can be
solved with filtering, removal of correlated variables.

\section{Software review}

Talk about what other implementations exist, specifically for \proglang{R}.

bootfs

leaps

DAAG

sampleSelection

\section[Introducing nestfs]{Introducing \pkg{nestfs}}

Talk about the general setup: iterations, inner folds, selection criteria,
termination criteria.

Talk about cross-validation and nested forward selection.

Talk about parallelisation options.

Talk about filtering.

\section{Practical examples}

Using some existing data sets, show how to use the package. Compare and
contrasts different options on the same data. 

Show what the package saves and reports.

\section{Conclusions}


\end{document}
